\documentclass[10pt]{article}

% Lines beginning with the percent sign are comments
% This file has been commented to help you understand more about LaTeX

% DO NOT EDIT THE LINES BETWEEN THE TWO LONG HORIZONTAL LINES

%---------------------------------------------------------------------------------------------------------

% Packages add extra functionality.
\usepackage{times,graphicx,epstopdf,fancyhdr,amsfonts,amsthm,amsmath,algorithm,algorithmic,xspace,hyperref}
\usepackage[left=1in,top=1in,right=1in,bottom=1in]{geometry}
\usepackage{sect sty}	%For centering section headings
\usepackage{enumerate}	%Allows more labeling options for enumerate environments 
\usepackage{epsfig}
\usepackage[space]{grffile}
\usepackage{booktabs}
\usepackage{forest}

% This will set LaTeX to look for figures in the same directory as the .tex file
\graphicspath{.} % The dot means current directory.

\pagestyle{fancy}

\lhead{Lab 9}
% \chead{Lab \Lab}
\rhead{\today}
\lfoot{CSCI 334: Principles of Programming Languages}
\cfoot{\thepage}
\rfoot{Spring 2022}

% Some commands for changing header and footer format
\renewcommand{\headrulewidth}{0.4pt}
\renewcommand{\headwidth}{\textwidth}
\renewcommand{\footrulewidth}{0.4pt}

% These let you use common environments
\newtheorem{claim}{Claim}
\newtheorem{definition}{Definition}
\newtheorem{theorem}{Theorem}
\newtheorem{lemma}{Lemma}
\newtheorem{observation}{Observation}
\newtheorem{question}{Question}

\setlength{\parindent}{0cm}


%---------------------------------------------------------------------------------------------------------

% DON'T CHANGE ANYTHING ABOVE HERE

% Edit below as instructed
\newcommand{\Lab}{8}	% Replace 0 with the actual problem set #
\newcommand{\ProblemHeader}	% Don't change this!

\begin{document}

\vspace{\baselineskip}
\begin{center}
    \textbf{Health Tracker Language Minimal Language}
\end{center}
\textbf{Minimal Former Grammar}
\begin{center}
    \begin{verbatim}
        <exp> ::= <day>␣<activity>
        <day> ::= date␣<month><day><year>
        <month> ::= 01 | 02 | 03 | 04 | 05 | 06 | 07 | 08 | 09 | 10 | 11 | 12
        <day> ::= 01| 02 | 03 | 04 | 05 | 06 | 07 | 08 | 09 | 10 | 11 | 12 
            | 13 | 14 | 15 | 16 | 17 | 18 | 19 | 20 | 21 | 22 | 23 | 24 | 25 |
            26 | 27 | 28 | 29 | 30 | 31
        <year> ::= <digit><digit><digit><digit>
        <digit> ::= 0 | 1 | 2 | 3 | 4 | 5 | 6 | 7 | 8 | 9
        <activity> ::= h2o␣<time> | h2o␣<time>␣<activity>
        <time> ::= <digit><digit><digit><digit>
        <h2o> ::= h2o␣<time>  
    \end{verbatim}
\end{center}
\textbf{Minimal Semantics}
\begin{center}
\scalebox{0.85} {
\begin{tabular}{ |c| |c| |c| |c|}
\hline
Syntax & Abstract Syntax & Type & Meaning \\
\hline
Time & float & float & Time is a primitive \\
\hline
h2o & Time of float & float & h2o is the time of which the h2o event takes place \\
\hline
Activity & h2o list & list of floats & Has every event which has taken place over the day of which you want to track \\
\hline
day & {date: Date; activity: Activity} & record of ints and float lists & keeps track of the date and list of the activities \\
\hline
\end{tabular} }
\newline
\newline
There is no precedence or associativity
\end{center}
\scalebox{0.4}{\includegraphics{images/watergraph.png}}

% DO NOT DELETE ANYTHING BELOW THIS LINE
\end{document}